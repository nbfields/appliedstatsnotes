%Preamble
%---
\documentclass{article}

%packages
%---
\usepackage{amsmath} %Advanced math typesetting
\usepackage[utf8]{inputenc} %Unicode support
\usepackage{hyperref} %Add a link
\usepackage{graphicx} %Add pictures
\usepackage{amssymb}
\usepackage{textcomp}
\usepackage{float}
\usepackage{wasysym}
\usepackage{listings}
\usepackage{wasysym}
\usepackage{systeme}

\graphicspath{{./images/}}

\hypersetup{colorlinks=true,
      linkcolor=blue,
      filecolor=magenta,
      urlcolor=cyan,
}
\begin{document}
  \section*{Deterministic vs Stochastic}
  \begin{itemize}
    \item Make sure to go over the typed notes too!
    \item Okay let's recap. We have:
      \begin{itemize}
        \item populations, population units
        \item Sample, subset of population
        \item Precise parameter, and statistic that estimates it
        \item Types of variables
        \item Statistical inferences like point estimation, confidence interval for range of values
        \item Hypothesis testing: Assume something is true, see how unlikely your predictor is, accept or reject
        \item If something lies outside of an $ n\% $ confidence interval that's equivalent to rejecting null hypothesis for that result
      \end{itemize}
    \item Notational convention:
      \begin{itemize}
        \item Underline: A vector, ex $  \underline{a},  \underline{b} $
        \item Uppercase: Matrix, ex $ A, B $
        \item hat: Statistic, eg  $ \hat{\theta} $
        \item $ \underline{a}^T, A^T $: Transpose
      \end{itemize}
    \item Unless otherwise specified, vectors are column vectors and transposes are row bois
    \item Ex 1: $ Y = x^2 $
    \item That's deterministic. Same $ x $, you get same $ Y $ value, over and over, no matter how many times you do it
    \item Ex 2:  $ Y = a + bx, a, b $ are known. $ a $ is intercept,  $ b $ is slope
    \item $ a $ is the default output if $ 0 $ is input, and $ b $ tells you how much the output changes when the input increases by one unit
    \item Important: When you talk about slope you \textit{increase} $ x $. It does \textbf{not} ``change''. It \textbf{increases}
    \item This straight line is also deterministic. AKA mathematical
    \item In general:
      \[
        Y = g(x), x \text{ is non-random }, Y \text{ is also non-random }
      \] 
    \item Problem: Particle boards are made. To examine this, boards produced at diff temperatures.\ Strength $ y $, temperature $ t $.
    \item Will points be exactly on a curve? Probably not. Some randomness/stochastic stuff will happen
    \item So you make a scatter plot
    \item In general:
      \[
        Y = g(x) + \epsilon, x  \text{ is non-random, } \epsilon \text{ is random error term, }, Y \text{ is therefore now random }
      \] 
    \item $ Y $ is the response/dependent variable, $ x $ is the predictor/regressor/independent variable/covariate
    \item Now: Impose assumptions on  $ \epsilon $:  They are IID with mean $ E(\epsilon) = 0 , V(\epsilon) = \sigma^2 < \infty $
    \item From that it follows:
      \[
        E(Y) = E(g(x)) + E(\epsilon) = g(x) + 0 = g(x)
      \] 
      \[
        V(Y) = \sigma^2 + V(g(x)) = \sigma^2 + 0 = \sigma^2
      \] 
    \item So what are we trying to find or achieve? $ g(x) $ I'm pretty sure
    \item Yup $ g(x) $
    \item Your mission, if you choose to accept it: Estimate $ g $
    \item Let's relax the formal definition
    \item aaaaaaaahhhh relaxing
    \item In general: We might have several regressors like a billion or so
       \[
        x_1, x_2, \ldots, x_k
      \] 
      \[
        Y = g(x_1, x_2, \ldots, x_k) + \epsilon
      \] 
    \item Simple case: Assume something about $ g $. The parametric form!
    \item Parametric form  $ \Rightarrow $ functional form of $ g $ is known to us, apart from particular parameters  $  \underline{ \beta } $
      \[
        Y = g(x_1, x_2, \ldots, x_k; \underline{beta}) + \epsilon
      \] 
    \item For instance $ g = a + bx $ where we don't know  $ a, b $
    \item More simple: Functional form of $ g $ is linear:
       \[
        Y = \beta_0 + \beta_1 x_1 + \beta_2 x_2 + \cdots + \beta_k x_k + \epsilon
      \] 
    \item That's the multiple linear regression model
    \item The unknown parameters are:
      \[
      \underline{\beta} = {(\beta_0, \beta_1, \ldots, \beta_k)}^T; \sigma^2
      \] 
      \[
        \Rightarrow (k + 2) \text{ unknown parameters }
      \] 
    \item We will work with this a lot
    \item The simplest case of multiple linear regression: $ k = 1 $
     \[
      Y = \beta_0 + \beta_1 x + \epsilon
    \] 
  \item Intercept, slope, error (with mean 0). So $ Y $ is an  $ RV, E(Y) = \beta_0 + \beta_1(x), V(Y) = \sigma^2 < \infty $
  \item This is the population regression model, and it's unknown to us because we don't know $ \underline{\beta} $ or $ \sigma $
  \item So take a sample out of the population and estimate the parameters, and then we can use our estimated regression model, we can do something
  \item So what are the  $ \beta $ boiz? The slope and the intercept? 
    \[
      \beta_0: \text{ Value of } E(Y) \text{ if } x = 0
    \] 
    \[
      \Rightarrow \text{ Mean of distribution of } Y \text{ when } x = 0
    \] 
    \[
      \beta_1: \text{ Amount of change in  } E(Y) \text{ by unit increase in } x
    \] 
  \item Always mention the mean of $ Y $ when thinking about these things!
  \item Sample regression model: 
    \[
      y_i = \beta_0 + \beta_1 x_i + \epsilon_i, i = 1, \ldots, n
    \] 
  \item Target: Estimate $ \beta_0, \beta_1, \sigma^2 $ based on sample data
  \item How? With least-squares ya dummy!!!!!!!!!!!!!!!!!!!!!!!!!!!!!!!1
  \end{itemize}
\end{document}
